\documentclass[FM,DP]{tulthesis}

\usepackage{polyglossia}
\setdefaultlanguage{czech}
\usepackage{fontspec}
\usepackage{xunicode}
\usepackage{xltxtra}
\setsansfont[Mapping=tex-text,BoldFont={* Bold},Numbers=OldStyle]{Myriad Pro}
\usepackage{hyperref}
\hypersetup{colorlinks=true, linkcolor=tul, urlcolor=tul, citecolor=tul}
\usepackage{graphicx}
\usepackage{listings}
\usepackage[toc,page]{appendix}
\usepackage{amsmath}
\usepackage{amssymb}
\newcommand{\argument}[1]{{\ttfamily\color{\tulcolor}#1}}
\newcommand{\prikaz}[1]{\argument{\textbackslash #1}}
\newenvironment{myquote}{\begin{list}{}{\setlength\leftmargin\parindent}\item[]}{\end{list}}
\newenvironment{listing}{\begin{myquote}\color{\tulcolor}}{\end{myquote}}
\sloppy


%%%%%%%%%%%%%%%%%%%%%%%%%%%%
\TULtitle{Vyhledávání jako služba}{Search as a Service}
\TULprogramme{Aplikovaná informatika}{Applied Informatics}
\TULbranch{Informační systémy a technologie}{Information Technologies}
\TULauthor{Bc. Luděk Veselý}
\TULsupervisor{Prof. Ing. Zdeněk Molnár, CSc.}
\TULyear{2017}


%%%%%%%%%%%%%%%%%%%%%%%%%%%%
\begin{document}
\ThesisStart{male}


%%%%%%%%%%%%%%%%%%%%%%%%%%%%
\begin{abstractCZ}
Tato diplomová práce popisuje návrh a tvorbu fulltextového vyhledávání poskytovaného jako služba.
\end{abstractCZ}

\begin{klicovaslovaCZ}
Fulltext, Elasticsearch
\end{klicovaslovaCZ}

\vspace{2cm}

\begin{abstractEN}
This diploma thesis describes creation of fulltext search service.
\end{abstractEN}

\begin{klicovaslovaEN}
Fulltext, Elasticsearch
\end{klicovaslovaEN}


%%%%%%%%%%%%%%%%%%%%%%%%%%%%
\begin{acknowledgement}
Rád bych poděkoval Ing. Ivanovi Jelínkovi za rady a pomoc při řešení.
\end{acknowledgement}


%%%%%%%%%%%%%%%%%%%%%%%%%%%%
\tableofcontents
\clearpage


%%%%%%%%%%%%%%%%%%%%%%%%%%%%
\begin{abbrList}
\textbf{ES} & Elasticsearch\\
\textbf{DBS} & Databázový systém\\
\end{abbrList}






%%%
1 Úvod    .
1.1 Cíle práce
Cíle práce jsou následující:
1.2 Cílová skupina
Cílovou skupinou této práce jsou 
1.3 Použité metody
1.4 Struktura práce
2 Komentovaná rešerše informačních zdrojů
3 Kapitoly vlastního textu
4 Závěr
4.1 Dosažení vytčených dílů
Bylo …
%%%



%%%%%%%%%%%%%%%%%%%%%%%%%%%%
\chapter{Úvod}

V~této diplomové práci práci popisuji postup nalezení veřejně přístupných zdrojů dat o~dopravních nehodách, stažení těchto dat, uložení dat ve vhodné databázi a možnost jejich využití v~dataminingové studii. Ministerstvo dopravy poskytuje od~roku 2006 v~rámci projektu Jednotná dopravní vektorová mapa \cite{jdvm} databázi dopravních nehod na~území ČR. Zde je možné nehody vyhledávat, zobrazit v~mapě a ke každé nehodě získat podrobný výpis. Data by však bylo možné využít efektivněji. Policie ČR sice pravidelně vytváří statistiky nehodovosti \cite{statistika-nehodovosti}, jde však pouze o~jednoduché promítnutí jednotlivých parametrů nehod do~grafu. Zpravidla se jedná o~zobrazení hodnot zkoumané veličiny v~daném období a porovnání s~předchozím obdobím:

\begin{figure}[h]
\center
\includegraphics[width=\textwidth]{nehody-dny.png}
\caption{Počty nehod v~jednotlivých dnech dle Policie ČR}
\label{foto}
\end{figure}

V~první fázi podrobíme data shlukové analýze. Nalezenými shluky mohou být například:

\begin{itemize}
\item 1. shluk
\begin{itemize}
\item Značka vozidla: Škoda, Ford
\item Charakteristika vlastníka vozidla: státní podnik, soukromá organizace
\item Kategorie řidiče: s~řidičským oprávněním skupiny B
\end{itemize}
\item 2. shluk
\begin{itemize}
\item Charakteristika vlastníka vozidla: Mezinárodní kamionová doprava
\end{itemize}
\end{itemize}

Nalezené shluky by poté byly rozděleny do~skupin dle následků odpovídajících nehod, tedy dle množství usmrcených osob, lehce zraněných osob, těžce zraněných osob, škody na~vozidle, celkové hmotné škody, případně úniku hmot.


%%%%%%%%%%%%%%%%%%%%%%%%%%%%
\chapter{Samotna prace...}

\section{Analyza problemu a pozadavku na aplikaci}

Vysvetleni problemu, ktery je resen
Specifika prostredi e-commerce, definice kontextu, ve kterem bude nastroj pouzivan
Definice pozadavku na samotne vyhledavani, akceptacni kriteria
Technicke pozadavky (datove formaty, rychlost odezvy...)
Popis oboru, kterych se prace dotyka (ziskavani informaci, NLP, big data...)
Analyza problemu fulltextoveho vyhledavani

\section{Fulltextove hledani obecne}

\section{Indexace}

Popis problemu indexovani dokumentu
Analyza dat pouzitych k vyhledavani
Zpracovani prirozeneho jazyka

Sklonovani, casovani, tvaroslovi obecne
Vaha slov, stop slova
Preklepy
Synonyma, zargon
Lemmatizace
Tezaurus
Navrh reseni

\section{Nastroje pouzitelne pro fulltextove vyhledavani}

MySQL
Elasticsearch
Sphinx
Algolia
Navrh architektury aplikace

Vyber technologii a nastroju
Definice pripadu uziti
Navrh architektury aplikace
Implementace aplikace

\section{Architektura aplikace}

Popis sluzeb, modulu, vztahu mezi nimi
Databaze

Nastaveni a nasazeni Elasticsearch
Implementace indexace
Implementace vyhledavani
Backend

Popis trid, rozhrani, implementacni detaily
Frontend

Navrh, komunikace s backendem
Deployment

CI, nasazovani na server

\section{Testovani, overeni}

Overeni funkcnosti a prinosu
Testovani, zda jsou splnena akceptacni kriteria
Porovnani s stavajicim resenim vyhledavani na konkretnim priklade
Diskuse mozneho budouciho rozsirovani




%%%%%%%%%%%%%%%%%%%%%%%%%%%%
\chapter{Závěr}

V této diplomové práci jsem nejprve nalezl zdroje dat...


%%%%%%%%%%%%%%%%%%%%%%%%%%%%
\begin{thebibliography}{Mm99}

\bibitem{strossa} STROSSA, Petr. \emph{Počítačové zpracování přirozeného jazyka}. \\
Vyd.~1. Praha: Oeconomica, 2011 000~s. ISBN 978-80-245-1777-3. % TODO: vydani a pocet stran

\bibitem{searching} AYSE Göker a DAVIES John. \emph{Information Retrieval: Searching in the 21st Century}. \\
Library of Congress Cataloging-in-Publication Data, 2009. ISBN: 978-0-470-02762-2. % TODO: vydani a pocet stran

\bibitem{mining} AGGARWAL Charu C., ZHAI ChengXiang. \emph{Mining Text Data}. \\
Springer New York Dordrecht Heidelberg London, 2000. ISBN 978-1-4614-3222-7. % TODO: vydani a pocet stran

\bibitem{es-guide} CLINTON Gormley, ZACHARY Tong. \emph{Elasticsearch: The Definitive Guide}. \\
O'Reilly Media, 2015. ISBN 978-1-4493-5854-9. % TODO: vydani a pocet stran

\bibitem{go-in-action} KENNEDY William. \emph{Go in Action}. \\
Manning Publications Co, 2016. ISBN 978-1-6172-9178-4. % TODO: vydani a pocet stran

\bibitem{elastic-reference} Elasticsearch. \emph{Elasticsearch Reference} [online]. \\
2017 [cit. 2017-03-03]. Dostupné z:\\
\url{https://www.elastic.co/guide/en/elasticsearch/reference/current/index.html}

\end{thebibliography}


%%%%%%%%%%%%%%%%%%%%%%%%%%%%
\appendix

\chapter{Obsah přiloženého DVD}

\begin{itemize}
\item Soubor Bakalarska\_prace\_2014\_Ludek\_Vesely.pdf
\begin{itemize}
	\item Text bakalářské práce
\end{itemize}

\end{itemize}

\chapter{Relační model databázového systému}

\begin{figure}[h]
\center
\includegraphics[width=0.86\textwidth]{relational.pdf}
\end{figure}


%%%%%%%%%%%%%%%%%%%%%%%%%%%%
\end{document}
